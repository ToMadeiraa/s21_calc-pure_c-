\documentclass[12pt, a4paper]{article}

\usepackage[utf8]{inputenc}
\usepackage{graphicx}

\title{Manual}

\author{by dianamab}

\date{Sep 18 2023}


\begin{document}

\maketitle

\pagebreak

\tableofcontents

\pagebreak

\section{Introduction}

The calculator has a graphical interface written using the gtk 3 library.
Calculator was developed on the Ubuntu 22.04.3 LTS.

An application with a graphical interface made on the GTK+ framework (C language). Calculator with the ability to set arbitrarily complex expressions, calculation algorithm
which is implemented using Reverse Polish Notation and stack. There is processing of all basic mathematical operations (logarithms, trigonometry). It is possible
to build graphs with a variable scale for different definition areas and value areas. Special \textbf{Credit} and \textbf{Deposit} calculators are made (like banki.ru and
calcus.ru).

Calculator does \textbf{not} process expressions in exponential form.

\pagebreak

\section{Functions and Operators}

The following mathematical operations and functions are implemented in the calculator:
  \begin{itemize}
    \item Arithmetic operators
    \begin{itemize}
      \item Brackets -- "( )"
      \item Addition -- "a + b"
      \item Substraction -- "a - b"
      \item Multiplication -- "a * b"
      \item Division -- "a / b"
      \item Power -- "a \^\ b"
      \item Modulus -- "a mod b"
      \item Unary plus -- "+a"
      \item Unary minus -- "-a"
    \end{itemize}
    \item Functions
    \begin{itemize}
      \item cos(x)
      \item sin(x)
      \item tan(x)
      \item acos(x)
      \item asin(x)
      \item atan(x)
      \item sqrt(x)
      \item ln(x)
      \item log(x)
    \end{itemize}
  \end{itemize}

These functions also are supported by the plotter.

\pagebreak

\section{Plotter}

The plotter supports all operators and functions from section 2.

The grapher has the ability to set the domain of definition and the range of value, but it does not have the ability to specify the number of points.

\end{document}
